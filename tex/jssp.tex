\section{Problema de sequenciamento de tarefas (\textit{Shop Scheduling}).}
O problema consiste em alocar da forma mais otimizada possível um conjunto de tarejas sobre um 
conjunto de máquinas de forma a minimizar o tempo total gasto, \textit{makespan}, para concluí-las.
Existem três categoria de problemas de sequenciamento:
\begin{itemize}
\item \textit{Job Shop:} cada tarefa tem sua própria sequencia de processamento passando pelas m máquinas.
\item \textit{Flow Shop:} todas as tarefas têm a mesma ordem de processamento.
\item \textit{Open Shop:} não há uma sequencia preestabelicida para o processamento das tarefas.
\end{itemize}
O trabalho aborda problemas de \textit{Flow Shop Scheduling} (FSSP).


\subsection{Problema Flow Shop Scheduling.}
As características do problema:
\begin{itemize}
\item Realizar $m$ tarefas: J1, J2, ... , Jm (produzir fios), sendo que:
\item Cada tarefa será processada em cada uma das $n$ máquinas: M1,..., Mn
\item O fluxo de processamento das $m$ tarefas nas $n$ máquinas é o mesmo para todas as tarefas.
\item Uma máquina processa apenas uma operação de cada vez, e não deve ser interrompida até sua conclusão.
\item São conhecidos os tempos de processamento de cada tarefa por máquina.
\end{itemize}
O objetivo é minimizar o tempo de conclusão de todas as tarefas (\textit{makespan}). O espaço de soluções
do problema é $m!$. Esta ordem de complexidade motiva o uso de meta-heurísticas para solucioná-lo.


